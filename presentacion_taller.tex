\documentclass{beamer}
\usetheme{Madrid}
\usepackage{graphicx}
\usepackage{hyperref}

\title{Juego de Preguntas Multijugador con IA y Firebase}
\author{Equipo: ErvinCaraval y colaboradores}
\date{Septiembre 2025}

\begin{document}

\begin{frame}
  \titlepage
\end{frame}

% Descripción General
\begin{frame}{Descripción General}
  \begin{itemize}
    \item MVP de un juego tipo “Preguntados” multijugador.
    \item IA genera y evalúa preguntas dinámicas.
    \item Firebase como base de datos no relacional.
  \end{itemize}
  \includegraphics[width=0.5\textwidth]{https://upload.wikimedia.org/wikipedia/commons/3/3a/Trivia_icon.png}
\end{frame}

% Propósito
\begin{frame}{Propósito}
  \begin{itemize}
    \item Gamificar el aprendizaje y el entretenimiento.
    \item Demostrar el uso de IA en desarrollo ágil con Scrum.
    \item Aplicar bases de datos no relacionales para escalabilidad.
  \end{itemize}
  \includegraphics[width=0.4\textwidth]{https://cdn-icons-png.flaticon.com/512/5968/5968945.png}
\end{frame}

% Audiencia Objetivo
\begin{frame}{Audiencia Objetivo}
  \begin{itemize}
    \item Estudiantes, familias y público general.
    \item Equipos educativos y docentes.
  \end{itemize}
  \includegraphics[width=0.4\textwidth]{https://cdn.pixabay.com/photo/2017/01/31/13/14/classroom-2021861_1280.png}
\end{frame}

% Funcionalidades Principales
\begin{frame}{Funcionalidades Principales}
  \begin{itemize}
    \item Registro y autenticación de usuarios.
    \item Creación y unión a partidas multijugador.
    \item Generación dinámica de preguntas usando IA.
    \item Almacenamiento en Firebase.
    \item Sistema de puntuación y ranking.
    \item Chat y retroalimentación en tiempo real.
    \item Panel de administración para preguntas.
    \item Historial de partidas y estadísticas personales.
  \end{itemize}
  \includegraphics[width=0.4\textwidth]{https://cdn-icons-png.flaticon.com/512/1055/1055687.png}
\end{frame}

% Historias de Usuario
\begin{frame}{Historias de Usuario (Ejemplos)}
  \begin{itemize}
    \item Como usuario nuevo, quiero registrarme para poder participar en partidas.
    \item Como jugador, quiero responder preguntas generadas por IA para probar mis conocimientos.
    \item Como administrador, quiero agregar/modificar preguntas para mantener el juego actualizado.
    \item Como jugador, quiero ver el resumen de la partida al finalizar.
  \end{itemize}
\end{frame}

% Product Backlog Inteligente
\begin{frame}{Product Backlog Inteligente}
  \begin{itemize}
    \item 15 historias de usuario, priorizadas y estimadas.
    \item Criterios de aceptación y pruebas sugeridas.
    \item Identificación de historias que usan IA.
    \item Ejemplo: “Generación dinámica de preguntas usando IA”.
  \end{itemize}
  \includegraphics[width=0.4\textwidth]{https://cdn-icons-png.flaticon.com/512/1055/1055687.png}
\end{frame}

% Release Plan
\begin{frame}{Release Plan}
  \begin{itemize}
    \item Sprint 1 (16-18 sept): Registro, autenticación y almacenamiento en Firebase.
    \item Sprint 2 (19-22 sept): Partidas multijugador y generación de preguntas con IA.
    \item Sprint 3 (23-25 sept): Panel de administración, puntuación y estadísticas.
    \item Sprint 4 (26-28 sept): Retroalimentación IA, filtrado inteligente y reportes.
    \item Sprint 5 (29-30 sept): Invitaciones, recuperación de contraseña, tiempo límite y resumen de partida.
    \item Entrega final: 30 de septiembre de 2025.
  \end{itemize}
  \includegraphics[width=0.4\textwidth]{https://cdn-icons-png.flaticon.com/512/5968/5968945.png}
\end{frame}

% Uso de Firebase
\begin{frame}{Uso de Firebase}
  \begin{itemize}
    \item Almacenamiento de usuarios, partidas y preguntas.
    \item Sincronización en tiempo real para partidas multijugador.
    \item Escalabilidad y seguridad.
    \item Integración sencilla con frontend y backend.
  \end{itemize}
  \includegraphics[width=0.4\textwidth]{https://firebase.google.com/images/brand-guidelines/logo-logomark.png}
\end{frame}

% Uso de IA
\begin{frame}{Uso de IA}
  \begin{itemize}
    \item Generación y refinamiento de preguntas.
    \item Retroalimentación automática sobre respuestas.
    \item Filtrado inteligente por categorías.
  \end{itemize}
  \includegraphics[width=0.4\textwidth]{https://cdn-icons-png.flaticon.com/512/4712/4712027.png}
\end{frame}

% Tecnologías Sugeridas
\begin{frame}{Tecnologías Sugeridas}
  \begin{itemize}
    \item Backend: Node.js, Python
    \item Base de datos: Firebase
    \item Frontend: React, Vue, Angular
    \item IA: OpenAI API, Gemini, Copilot
  \end{itemize}
  \includegraphics[width=0.4\textwidth]{https://cdn-icons-png.flaticon.com/512/5968/5968945.png}
\end{frame}

% Estructura Recomendada
\begin{frame}{Estructura Recomendada}
  \begin{itemize}
    \item src/: código fuente
    \item tests/: pruebas automatizadas
    \item docs/: documentación y prompts
  \end{itemize}
\end{frame}

% Conclusión
\begin{frame}{Conclusión}
  \begin{itemize}
    \item Proyecto alineado con Scrum y uso de IA.
    \item Backlog y plan de trabajo claros.
    \item Listos para iniciar el desarrollo en equipo.
  \end{itemize}
  \includegraphics[width=0.4\textwidth]{https://cdn-icons-png.flaticon.com/512/1055/1055687.png}
\end{frame}

\end{document}
